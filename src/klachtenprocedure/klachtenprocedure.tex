
\documentclass[a4paper]{article}
\usepackage[a4paper,margin=2cm]{geometry}
\usepackage{parskip}

\raggedright
\begin{document}

\title{Klachtenprocedure}
\date{}
\maketitle

\section*{Inleiding}

Roodwitblauw stelt er prijs op wanneer opdrachtgevers, docenten of cursisten hun mening kenbaar maken over verbeterpunten en onvolkomenheden in de organisatie. Daarvan kan de organisatie leren en zij kan bezien of werkwijze, structuur of beleid aangepast moeten worden om dergelijke onwenselijkheden te voorkomen.

Wanneer we het in deze regeling over klachten hebben, gaat het niet over dat soort verbeterpunten en onvolkomenheden, maar het gaat het om gedragingen en beslissingen (of de afwezigheid daarvan) die het belang van de ander schaden.

\section*{Wie kunnen klagen?}

Opdrachtgevers, (ex-)cursisten, begeleiders van (ex-)cursisten (verder: ``klager'') kunnen een klacht indienen over gedragingen en beslissingen van medewerkers van Roodwitblauw dan wel over het nalaten van gedragingen en het niet nemen van beslissingen door medewerkers van Roodwitblauw.

Gedacht kan worden aan klachten over:

\begin{itemize}
	\item geleverde diensten van Roodwitblauw
	\item de werkwijze van de docent in bv. de omgang met cursist(en)
	\item het niet nakomen van afspraken
\end{itemize}

\section*{Bij wie kunnen klagers terecht?}

Klager richt zich in eerste instantie tot degene door wie de klager zich benadeeld voelt. De betrokken medewerker van Roodwitblauw probeert de klacht in overleg met de klager op een voor beide bevredigende wijze op te lossen.

Lukt het niet in een gesprek tot een bevredigende oplossing te komen, dan kan de klager terecht bij Invitaal Taaltrainingen.

Een klacht dient schriftelijk te worden ingediend. De klacht bevat ten minste:

\begin{itemize}
	\item de naam en het adres van de klager
	\item de cursus die klager volgt
	\item een omschrijving van de klacht
	\item dagtekening
\end{itemize}

U kunt uw klacht sturen naar mw. M.J. de Boer,  Hollandseweg 358, 6705 BD Wageningen. De ontvangst van de klacht wordt schriftelijk bevestigd aan de klager. De beklaagde heeft 2 weken de tijd om haar standpunt toe te lichten. Dat standpunt wordt aan de klager meegedeeld, waarna de klager 2 weken de tijd heeft hierop te reageren. Binnen twee weken volgt een schriftelijke uitspraak van de klachtencommissie. Deze uitspraak is bindend.

\section*{Geheimhouding}

Iedereen die betrokken is bij de uitvoering van een klachtenprocedure en daarmee beschikking krijgt over gegevens die vertrouwelijk zijn, is verplicht tot geheimhouding daarvan.

\section*{Registratie}

Per kalenderjaar worden klachten geregistreerd en bewaard in een klachtendossier. De klachten worden jaarlijks ge\"evalueerd.

Na twee jaar wordt een klachtendossier vernietigd.

\end{document}
