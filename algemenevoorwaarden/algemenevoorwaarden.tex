
\documentclass[a4paper]{article}
\usepackage[a4paper,margin=2cm]{geometry}
\usepackage{parskip}

\renewcommand{\theenumi}{\alph{enumi})}
\renewcommand{\labelenumi}{\theenumi}

\renewcommand{\thesection}{Artikel \arabic{section}.}

\raggedright
\begin{document}

\title{Algemene Voorwaarden}
\date{}
\maketitle

\section{Algemene bepalingen}

\begin{enumerate}
	\item Deze algemene voorwaarden zijn van toepassing op alle aanbiedingen, opdrachten en overeenkomsten van Roodwitblauw.
	\item Gehele of gedeeltelijke afwijking van deze algemene voorwaarden is slechts mogelijk indien en voor zover schriftelijk overeengekomen.
	\item In deze algemene voorwaarden worden de hierna volgende termen in de navolgende betekenis gebruikt:
	\begin{itemize}
		\item ``opdrachtnemer'' –-- De gebruiker van deze voorwaarden: Roodwitblauw (RWB).
		\item ``opdrachtgever'' –-- De natuurlijke of rechtspersoon met wie opdrachtnemer een overeenkomst sluit en/of aan wie opdrachtnemer een aanbod doet.
		\item ``de overeenkomst'' --– De overeenkomst van opdracht tussen opdrachtnemer en de opdrachtgever.
	\end{itemize}
\end{enumerate}

\section{Algemene gegevens}

\begin{enumerate}
	\item Roodwitblauw  biedt cursussen ‘Nederlands als tweede taal’ aan. De inhoud, de lengte en de frequentie van de cursus worden in overleg tussen RWB en de opdrachtgever vastgesteld.
	\item De lessen zullen worden gegeven op het kantoor van RWB: de Hoef 27, 6708 DB Wageningen, tenzij anders is overeengekomen.
\end{enumerate}

\section{Overeenkomst}

\begin{enumerate}
	\item Elke opdrachtgever heeft recht op een gratis intakegesprek van 30 minuten. Wanneer opdrachtgever lessen wil gaan volgen bij RWB, wordt een contract opgesteld en ondertekend. De opdrachtgever heeft het recht om binnen 10 dagen na ondertekening van het contract, dit te herroepen.
\end{enumerate}

\section{Be\"eindiging overeenkomst}

\begin{enumerate}
	\item De opdrachtgever kan de overeenkomst te allen tijde annuleren, mondeling of schriftelijk, zij het langer dan 24 uur voor de eerstvolgende les.
	\item De door opdrachtnemer gemaakte kosten en reeds aangegane verplichtingen die zij niet meer kan annuleren, komen bij annulering door de opdrachtgever volledig voor rekening van de opdrachtgever.
	\item Opdrachtnemer behoudt zich het recht voor, in geval van bijzondere omstandigheden, cursussen of cursusbijeenkomsten te annuleren of onderbreken, zonder enige verplichting tot schadevergoeding.
	\item In geval van annulering van een cursusbijeenkomst door de opdrachtnemer, bijvoorbeeld door plotselinge ziekte docent, zal de opdrachtnemer een grote inspanning leveren om de betreffende bijeenkomst alsnog op een in overleg te bepalen tijdstip te laten plaatsvinden. De opdrachtgever werkt hieraan mee. De door de opdrachtnemer geannuleerde les zal niet in rekening worden gebracht.
\end{enumerate}

\section{Facturatie en betaling}

\begin{enumerate}
	\item Indien gewenst en overeengekomen kan de opdrachtgever na elke les contant betalen. Hij/zij ontvangt hiervoor een kwitantie.
	\item Indien gewenst en overeengekomen wordt maandelijks een factuur verstuurd. De betaling van de door opdrachtnemer verzonden factuur dient binnen 10 dagen na de factuurdatum, zonder enige aftrek of verrekening, door de opdrachtgever te worden voldaan.
	\item Wanneer betaling van een door opdrachtnemer aan opdrachtgever toegezonden factuur niet heeft plaatsgevonden binnen de in de voorwaarden voorgeschreven betalingstermijn, wordt de opdrachtgever geacht in verzuim te zijn zonder dat enige ingebrekestelling zal zijn vereist. Opdrachtnemer is in dat geval gerechtigd om administratiekosten ten bedrage van 2\% van het factuurbedrag of het onbetaalde gedeelte daarvan, bij de opdrachtgever in rekening te brengen.
	\item Indien de opdrachtgever (een deel van) zijn verplichtingen niet nakomt dan komen alle (buiten)gerechtelijke kosten teneinde nakoming van de verplichting van de opdrachtgever te bewerkstelligen ten laste van de opdrachtgever.
\end{enumerate}

\section{Aansprakelijkheid}

\begin{enumerate}
	\item De opdrachtnemer is aansprakelijk voor tekortkomingen in de uitvoering van de opdracht, voor zover deze het gevolg zijn van het niet in acht nemen door de opdrachtnemer van zorgvuldigheid, deskundigheid en het vakmanschap waarop bij het uitbrengen van de adviezen in het kader van de betrokken opdracht mag worden vertrouwd.
	\item De aansprakelijkheid voor de schade veroorzaakt door de tekortkomingen wordt beperkt tot het bedrag dat de opdrachtnemer voor zijn werkzaamheden in het kader van die opdracht heeft ontvangen.
	\item Met inachtneming van de voorgaande bepaling reikt de aansprakelijkheid van opdrachtnemer nimmer verder dan de dekking van de door opdrachtnemer afgesloten aansprakelijkheidsverzekering.
	\item In geval van gebruikmaking door opdrachtnemer van een cursus/trainingsruimte van de opdrachtgever is alle aansprakelijkheid die verband heeft met de cursus/trainingsruimte zoals voor (brand)veiligheid, voor de opdrachtgever.
\end{enumerate}

\section{Auteursrecht}

\begin{enumerate}
	\item Het auteursrecht op door opdrachtnemer uitgegeven brochures, cursusbeschrijvingen, cursusmateriaal, trainingsmateriaal, rapportagemateriaal, evaluatiemateriaal en andere documenten berust bij de opdrachtnemer, tenzij de naam van een andere rechthebbende op het werk zelf is vermeld. Zonder schriftelijke toestemming van opdrachtnemer is het de opdrachtgever, deelnemer en/of derden niet toegestaan bedoelde documenten te publiceren of op welke wijze dan ook te vermenigvuldigen.
\end{enumerate}

\section{Vertrouwelijkheid}

\begin{enumerate}
	\item De opdrachtnemer verplicht zich om alle cursistgegevens vertrouwelijk te behandelen.
	\item De opdrachtgever is verplicht tot geheimhouding van alle informatie en gegevens van de opdrachtnemer jegens derden. De opdrachtgever zal zonder toestemming van opdrachtnemer aan derden geen mededeling doen over de aanpak van opdrachtnemer, haar werkwijze en dergelijke, dan wel haar rapportage ter beschikking stellen.
\end{enumerate}

\section{Toepasselijk recht}

\begin{enumerate}
	\item Op alle onder deze voorwaarden gedane aanbiedingen of gesloten overeenkomsten is het Nederlandse recht van toepassing.
\end{enumerate}

\end{document}
